%mainfile: master.tex
\chapter{Indledning}
\label{chap:Indledning}

\section{Nytorv før 00'erne}
\label{sec:nytorv_foer_null}
Aalborg Nytorv er et af samlingspunkterne i Aalborg , og der færdes mange mennesker og køretøjer dernede. Bevirker det, at cyklister og fodgængere føler sig utrygge dernede? Og hvordan kan man i det hele taget modvirke, at der sker konflikter mellem de forskellige trafikantgrupper på Nytorv? Det er spørgsmål, som bliver undersøgt i denne rapport.
Tilbage i 1980 vedtog man i Aalborg kommune en arealanvendelsesplan for Aalborg Nytorv.
%(http://apps.aalborgkommune.dk/images/teknisk/PLANBYG/lokplan/01/10-010.pdf (s. 2)) 
Formålet med denne plan, var at gøre Nytorv til et såkaldt ”bybånd”, altså en bykerne med et godt trafik flow, med særlig hensigt på en kollektive trafik. Man ville etablere et buscenter, som skulle strække sig fra omkring det nuværende Algade ned til Nytorv og op til Slotsgade, hvor der så kun måtte køre busser. Det skulle gøre det nemmere for busserne, at komme hurtigere frem i trafikken. Det  væsentlige hovedformål, taget ud fra en redegørelse af den daværende trafikplan, var :
”at der skabes bedre omstigningsmuligheder, dels mellem bybusserne     indbyrdes, dels mellem bybusser og rutebiler.” \cite{datasheet_pir1}
\begin{figure}
\centering
\includegraphics[scale=0.5]{opserveringspunkter2.PNG}
\caption{Kort over Aalborg Midtby}
\end{figure}
%(http://apps.aalborgkommune.dk/images/teknisk/PLANBYG/lokplan/01/10-010.pdf (s. 3))
Det var altså en plan for at gøre det lette tilgængeligt for bybusserne at transportere sig rundt i bymidten, uden at skulle konflikteres med den individuelle trafik. Dermed også gøre det hurtigere som borger, at transportere sig rundt i Aalborg.
I 1998 sker der så en omlægning af Østerågade/Nytorv. 
%http://www.trafikdage.dk/td/papers/papers00/Dag1/paper/1203.pdf (s. 85)) 
Man omlagde Nytorv således, at man gav den en granitbelægning, som var ens både på vejen og fortovet. Det kan også ses dernede den dag i dag, at det er meget svært at skelne mellem fortov og vej. Man gjorde også vejbredden mindre, og man fjernede til og med det lyskryds, som var på Nytorv. Man gik altså fra et mere opdelt fortov/trafikvej, hen mod et såkaldt ”Sharedspace”. Begrebet Sharedspace vil blive beskrevet mere detaljeret senere i rapporten. En af ændringerne fra trafikplanen i 1980 var, at nu måtte kun privatbiler med ærinder køre inde på Nytorv. Derudover busser og erhvervsbiler. Adgang for biler på Nytorv var altså nu forbudt. Udfaldet af dette var et fald af den private bilkørsel på Nytorv med ca. 60%. 
%(Der er en tabel i et link Bjørn har, som viser det)
Meningen med at forhindre den private bilkørsel, var at gøre Aalborg bymidte til et sted, hvor der var gode vilkår for cyklister, fodgængere og den kollektive trafik. Det skulle være lettere at transportere sig rundt på cykel og gåben, da Nytorv nu var et fællesareal, hvor det udelukkende var cykler, fodgængere og busser der færdes.

\section{Hvorfor færdes der mange mennesker på Nytorv?}
\label{sec:hvorfor_faerdes_der_mange_mennesker_paa_nytorv?}
Det er interresant at se på hvorfor der færdes så mange mennesker på Nytorv/Østerågade, som der gør, for at få en forståelse af, hvorfor der til tider kan være meget trafik dernede, og derved for nogle en give en utryghed i at færdes dernede på gåben.
Aalborg er med sine 109.092 indbyggere i 2014 Danmarks 4 største by.%(http://www.denstoredanske.dk/Danmarks_geografi_og_historie/Danmarks_geografi/Jylland/Jylland_-_byer/Aalborg Aalborg kommune) 
Det er især på Nytorv/Østerågade, at der færdes mange mennesker, som enten transportere sig med bus, på cykel eller er gående. Den store tiltrækning på Nytorv ligger i dels de mange forskellige offentlige og kommercielle servicefunktioner, og dels de mange shopping- og cafémuligheder. 
%(http://www.aalborgkommuneplan.dk/kommuneplanrammer/midtbyen/aalborg-midtby/default.aspx) 
Et af projekterne er Aalborgs havnefront, som i flere år har været under omdannelse. Således er havnen gået fra at være industri arbejdsplads, til en af byens attraktioner. Havnefronten tilbyder arkitektur, gastronomi og gode faciliteter til at samles og hygge med venner.
%(http://www.visitaalborg.dk/aalborg/aalborgs-havnefront) 
Shoppingmuligheder er der som sagt også mange af. Her finder man de 2 store gågader, Bispensgade og Algade, hvor forskellige mode- og specialforretninger ligger side om side. Gågaderne er adskilt af Østerågade, hvorfra Nytorv også ligger, som tilbyder andre shoppingmuligheder med Salling, Friis Citycenter og Føtex. (((Referer til kortet)))
%( http://www.visitdenmark.dk/da/nordjylland/shopping/shopping-i-aalborg) 
Nytorv og Østerågade ligger derfor meget centreret i Aalborg, og er et stort samlingspunkt for beboerne i Aalborg og omegn. Østerågade ligger i forlængelse af Boulevarden og strækker sig til Toldbod Pl., som er forbundet med strandvejen, hvor havnefronten ligger. På grund af de mange shoppingmuligheder og de fælles samlingspunkter, kan der på nogle tidspunkter af dagen være rigtig mange trafikanter, som færdes nede på Nytorv. Ligeledes færdes der også mange trafikanter, som cykler eller pendler til og fra arbejde hver dag, som på nogle tidspunkter af dagen også kan skabe et højt trafikflow. Derfor er det interresant at se på, om fodgængerne på Nytorv føler sig trygge, når der er mange cykler, busser og for den sags skyld også biler, som færdes dernede. For Østerågade og Nytorv er ikke kun et knudepunkt for fodgængere. Området bliver også brugt som busholdeplads for de mange by- og metrobusser som kører igennem hver time. Hele 32 busser med forskellige destinationer har stoppested i området. Og selvom at det er gjort ulovligt, for privat billister at kører på størstedelen af Østerågade, kører der stadig 3000 bilister igennem hver dag.(((Kilde: selve den udleverede opgave)))  På Nytorv ser man en fællesbelægning, både på fortov og vej. (((referer til et billede))) Der er heller ikke meget afmærkning, og bliver benyttet af trafikanterne som et slags fællesareal. Man kunne måske argumentere for, at Nytorv bliver brugt som et slags Shared Space, som vil blive beskrevet i det følgende. 
\section{Shared Space}
\label{sec:shared_space}
Hvad er shared space
Shared space er et begreb, der definerer en vejledning til trafik- og byplanlæggere. Vejledningen giver anbefalinger til vejmyndigheder, rådgivere og arkitekter, som indeholder anbefalinger til, hvordan shared space burde anvendes, skiltes og udformes, som en indretning af gaderum i de danske byer. Derudover indeholder vejledningen anbefaler til, hvilke trafikmængder og hastigheder der bør være i et shared space område.
Fremføring til shared space
Shared space konceptet udkom i forbindelse med en række hollandske projekter fra 1990’erne, begyndelsen af 2000’erne og i forbindelse med en EU’s interReg samarbejde i perioden 2004-2008.
Shared space indebærer, at de forskellige trafikantgrupper så som den kollektive trafik, privatbiler, parkering, godstrafik, cykelister og fodgængere deler det offentlige rum med hinanden, uden at nogle grupper burde være dominerende. Shared space konceptet er for det offentlige byrum en prioritering af det sociale liv for mennesker. Derudover stræber det efter at skabe velfungerende og multifunktionelle byrum, hvor alle trafikantgrupper er ligeværdige i trafikken. Alle trafikantgrupper integreres og færdes på samme areal. Her tilpasser trafikanternes adfærd til den sociale adfærd, som de mennesker, der opholder sig i udviser. Shared space’s fysiske udformning er et område, hvor der ikke er nogle traditionel opdeling for fodgængere, cyklister og kørertøjer. Her er der et minimum af skiltninger og afmærkninger.

Historisk tilbageblik på shared space
Der var et fredeligt samspil mellem trafikantgrupperne, hvor byrummene typisk fungerede som shared space, indtil massebilismen indtog, og her blev trafikantgrupperne blev differentieret. Den hurtige trafik skulle skilles fra den langsomme, og den hårde trafik fra den bløde. Der blev etableret separate cykel- og gangstier på de større veje. Grundidéen blev udviklet med specielt henblik på de mange nye byområder og var et udtryk for ønsket om, at sikre både biltrafikkens sikkerhed og skabe trafiksikre boligområder. Der blev etableret en lang række trafikdifferentierede byområder både i Danmark og udlandet i perioden 1960-1975.
I begyndelsen af 1970’erne blev integrationen mellem de bløde og hårde trafikanter første gang genintroduceret i Holland, som en modreaktion på differentiering af trafikanterne i byområderne. På derværende tidspunkt blev der i hollandske boligområder etableret ”woonerven” (bebolig gader). Ønsket var grundlæggende at reducere antallet af kørende og deres hastighed. Det foregik gennem etablering af fysiske ændringer såsom bump, forsætninger og indsnævringer af kørebanearealet, samt ved bevidst brug af beplantning og pynte området. Formålet med de fysiske ændringer var for, at lade bilerne køre i gaderne, men med meget lave hastigheder, for at tage hensyn til fodgænger og legende børns betingelser.  Det resulterede en lang række kopieringer og fortolkninger af de hollandske initiativer, især i Nordeuropa eksempelvis i Home Zone i Storbritannien, Spielstrassen i Tyskland, gårdsgader i Danmark og Sverige i opholds- og legeområder.
Byer med gader og torve har altid indeholdt flere funktioner.  Byrummene har blandt andet den funktion, at den burde kunne afvikle trafikken bestående af kollektiv trafik, privatbiler, parkering, godstrafik, cykelister og fodgængere. Derudover har byrummene også til funktion, at blive brugt til ophold og handel for byens beboere og besøgende.
Integration af både forskellige trafikantgrupper og opholdsfunktioner kan opleves nogle steder i vores byer og andre steder i Nordeuropa. Denne trafikalblanding har gennem tiderne blevet kaldt mange forskellige ting. Siden starten af 1980’erne er der især i Nordeuropa blevet udformet en lang række forskellige nye blandede trafik- og opholdsrum.
Shared space konceptet udkom i forbindelse med en række hollandske projekter fra 1990’erne, og i begyndelsen af 2000’erne, som var i forbindelse med EU’s interReg III samarbejde i perioden 2004-2008, hvor den tideligere Ejby Kommune på Fyn deltog.

Andre typergader
Gågader
I begyndelsen af 1960’erne blev en række tideligere gader ændret til gågader. Det typiske ved gågader er, at de ofte ligger centralt ved beliggende butiks- eller strøggader, som primært er indrettet til gående, hvor biltrafik er fjernet med undtagelse af varekørsel. Derfor er gågader kendetegnet ved, at være trafikale differentierede gadetyper.
Gågaders karakteristiske fysiske udformning er en indretning, hvor belægningen fremtræder forskelligt fra traditionelle kørearealer. For at få gågaden til at fremstå sammenhængende fra facade til facade, bliver niveauforskellene mellem fortov og vej fjernet. I andre sammenhænge etableres gågader, hvor kørselstilladelse, som er eksisterende for den bevidste udformede sivegade.
Sivegader
I midten af 1980’erne begyndte man med, at ombygge mere centrale beliggende by-gader og handelsstrøg med større trafikmængder til sivegaderne, som er nogle stilleveje der ikke er defineret som et begreb i færdselsloven. Det karakteristiske ved sivegader er, at de er opbygget med smalle kørebaner uden niveauforskelle til de bagvedliggende arealer, og gaden fremstår oftest som en sammenhængende flade, med markeringer af parkerings- eller udstillings-/opholdsarealer. Helhedsvurderingen er en multifunktionel by-gade og ikke en trafikvej, som er understreget i den fysiske anvendelse af belægninger og belysning, det kendetegnes fra gågader, som i højere grad er afgørende for opfattelsen af byrummets karakter og den ønskede trafikale facon. I sivegader er der oftest skiltet med C55 lokal hastighedsbegrænsninger til 20 km/t eller 30 km/t, og E53 områder med fartdæmpning eller med andre tilfælde E49 gågade med kørsel tiltalt som undertavle.
Andre typer gader sammenlignet med shared space
Sivegader og andre forskellige gadetyper kan på mange måder ses som shared space- ligende gaderum, men alligevel adskiller de sig fra shared space. Shared space anvendes primært i centrale byområder, hvor der er høj trafikmængder, her er hensigten at trafikantgrupperne skal kunne færdes under fælles hensyntagen, hvorimod at andre gadetyper anvendes i primært boligområder med meget lav trafikmængder, hvor hensigten er at skabe mulighed for leg og ophold i vejens bredde.
Fælles for gågader og shared space områder er at de begge er beliggende i centrale byområder. Forskellen mellem gågader og shared space er mængden af trafikanterne i områderne. I gågader er mængden af kørende trafikanter mindre end i shared space, hvor kørsel er tilladt. Kørende trafik færdes under de gåendes betingelser, det vil sige, at de gående er i overtal og prioriteret frem for kørende trafik, da det er en gade for gående. I shared space områder er mængden kørende trafikanter højere end i gågader, her er ingen trafikantgrupper prioriteret frem for andre.
Sivegader minder allermest om shared space områder, da der er en multifunktionel indretning, som ligger til grund for strukturen og udformningen, hvor de forskellige trafikantgrupper færdes ligeværdigt i hele områdets udtrækning og uden den traditionelle fysiske opdeling i gang/opholds- og kørearealer. Skiltning af sivegader er ikke tydelige, da begrebet ikke findes som begreb i færdselsloven, derfor skiltes den forskelligt fra sted til sted.


\section{Interviews af fodgængere}
\label{sec:interviews_af_fger}
Kort om det kvalitative interview
Det kvalitative interview virker ofte som en metode for, at forbedre nogle bestemte forholde. Da det kvalitative interviews styrke og formål, handler om at finde kernen af kendte og ukendte ønsker og behov i forskellige livsfaser og situationer.
Det kvalitative interview kan sammenlignes med den kvantitative metode, hvor de er hinandens modsætninger. Den kvantitative metode anvendes for at opnå en bred forståelse, holdning eller kendskab til et produkt, problem eller område.  Formålet med den kvantitative interview er at klargøre problemets størrelse, altså at opnå en generel oversigt over de medvirkende interviewers forståelse, holdning eller kendskab til det bestemte produkt, problem eller område. Her bliver der bl.a. brugt spørgeskemaer, personlige interviews eller telefoninterviews. Hvorimod den kvalitative metode søger mere i dybden af det bestemte område. Den kvalitative metode forsøger at opnå en dybere forståelse af de forhold den interviewet udtaler sig om i interviewet. Den forsøger ikke bare at generaliser bestemte forholde, men derimod fås flere aspekter med. Derudover har fortællingen i det kvalitative interview forrang, hvilket betyder at intervieweren hele tiden skal finde flere holdninger, forståelser eller kendskaber til produktet, det gøres ved konstant at spørge ind til oplevelsen eller lign., hvor intervieweren giver plads til hverdagssproget og hverdagsbeskrivelser. Her kan der bl.a. anvendes dybdeinterviewer, hvor der tales med en person ad gangen.
Formålet med vores interview
Formålet med vores interview er, at finde initierende problemer og løsninger til, hvordan vi kan opnå en effektiv og innovativ sammenbinding af gågade systemet i Aalborg Nytorv/Østerågade. Der blev interviewet fodgængere og cykelister, med hensigten om at de skulle have medbestemmelse i, at skabe en bedre oplevelse af sikkerhed og tryghed i området. I den forbindelse gennemførte vi ca. 20 dybdeinterviewer med forbipasserende fodgængere og cyklister.
Vores interview resultater
Ud fra vores interviews kom vi frem til, at der var en større enighed om hvorvidt forbipasserende fodgænger og cyklister følte sig utrygge ved at passere Nytorv/Østerågade området i forbindelse med, det integreret trafikmiljø der er. Derudover var der også enighed iblandt de forbipasserende fodgængerne og cyklisterne om, at der ønskes en differentiering mellem cyklister og kørebanen, derudover var der også flere ønsker om en køretøjsfrizone.
