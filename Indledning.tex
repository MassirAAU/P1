%mainfile: master.tex
\chapter{Indledning}
\label{chap:Indledning}

\section{Nytorv før 00'erne}
\label{sec:Nytorv før 00'erne}
Aalborg Nytorv er et af samlingspunkterne i Aalborg , og der færdes mange mennesker og køretøjer dernede. Bevirker det, at cyklister og fodgængere føler sig utrygge dernede? Og hvordan kan man i det hele taget modvirke, at der sker konflikter mellem de forskellige trafikantgrupper på Nytorv? Det er spørgsmål, som bliver undersøgt i denne rapport. 
Tilbage i 1980 vedtog man i Aalborg kommune en arealanvendelsesplan for Aalborg Nytorv . Formålet med denne plan, var at gøre Nytorv til et såkaldt ”bybånd”, altså en bykerne med et godt trafik flow, med særlig hensigt på en kollektive trafik. Man ville etablere et buscenter, som skulle strække sig fra omkring det nuværende Adelgade ned til Nytorv og op til Slotsgade, hvor der så kun måtte køre busser. Det skulle gøre det nemmere for busserne, at komme hurtigere frem i trafikken. Det  væsentlige hovedformål, taget ud fra en redegørelse af den daværende trafikplan, var :
”at der skabes bedre omstigningsmuligheder, dels mellem bybusserne     indbyrdes, dels mellem bybusser og rutebiler.”    
Det var altså en plan for at gøre det lette tilgængeligt for bybusserne at transportere sig rundt i bymidten, uden at skulle konflikteres med den individuelle trafik. Dermed også gøre det hurtigere som borger, at transportere sig rundt i Aalborg. 
I 1998 sker der så en omlægning af Østerågade/Nytorv, som følge af JUPITER 2 projektet, som var en miljøplan vedtaget i EU, og havde til formål at mindske miljøgenerne i byområder. Med i denne plan var også andre store byer som Firenze, Bilbao, Liverpool og Gent . Man omlagde Nytorv således, at man gav den en granitbelægning, som var ens både på vejen og fortovet. Det kan også ses dernede den dag i dag, at det er meget svært at skelne mellem fortov og vej. Man gjorde også vejbredden mindre, og man fjernede til og med det lyskryds, som var på Nytorv. Man gik altså fra et mere opdelt fortov/trafikvej, hen mod et såkaldt ”Sharedspace”. Begrebet Sharedspace vil blive beskrevet mere detaljeret senere i rapporten. En af ændringerne fra trafikplanen i 1980 var, at nu måtte kun privatbiler med ærinder køre inde på Nytorv. Derudover busser og erhvervsbiler. Adgang for privatbiler på Nytorv, var altså nu forbudt. Udfaldet af dette var et fald af den private bilkørsel på Nytorv med cirka 60 procent . Meningen med at forhindre den private bilkørsel, var at gøre Aalborg bymidte til et sted, hvor der var gode vilkår for cyklister, fodgængere og den kollektive trafik. Det skulle være lettere at transportere sig rundt på cykel og gåben, da Nytorv nu var et fællesareal.

\section{Generelt om Nytorv}
\label{sec:Generelt om Nytorv}
Det er fatimah haha
\section{Shared Space}
\label{sec:Shared Space}
Hej mit navn \cite {datasheet_pir1}
\section{Første Oberservation}
\label{sec:Første Oberservation}

\section{Interviews af fodgængere}
\label{sec:Interviews af fodgængere}
