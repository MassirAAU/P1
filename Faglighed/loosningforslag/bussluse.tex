\subsection{Bussluse}
\label{bussluse}

En bus sluse er en trafikanordning, som har til formål at lade busser, men ikke andre trafikanter, køre på en vejstrækning. 
I Aalborg har man tidligere anvendt bus sluser på mange forskellige steder, som ved skoler og områder med stilleveje. På Nytorv/Østerågade ligger mange restauranter, shoppingfaciliteter og cafeér, og som nævnt tidligere, er der mange mennesker, som færdes i området. Samtidig er der forbud af gennemkørsel af privatbiler i området, men det er aldrig blevet respekteret fuldt ud, da mere end 3000 biler passerer hverdag Østerågade. ((kilde))
Projektet undersøger hvordan bus sluserne kan forhindre gennemkørslen af privatbiler, og derved være en løsning på en af trafikkonflikterne i området. Forslaget har til formål, at  skabe en sikkerhed, som kunne give mennesker tryghed, når de besøger på Nytorv/Østerågade.      
Fordele: Bus sluserne kan forhindre gennemkørsel af privatbiler, således kun busser og varebiler kan køre igennem, og hermed reducerer trafikken i område. Det vil give et tryggere område for cyklisterne og fodgængerne. Byggeriet af bus sluserne, er billigere end andre vejarbejder projekter, og byggeprojektet bruger ikke så lange tid.   
Ulemper: Danmark har ofte dårlige vejr, der sner, regner og blæser. Efter det bliver bus sluserne beskidte, hvilket ikke ser pænt ud. Hvis de ældre borgere og børn vil over gaden, er det ikke nemt at gå over bus sluserne. Og hvis der sker noget på Nytorv/Østerågade som kriminalitet eller nogen mennesker pludselig bliver meget syg, kan det også blive meget svært for politibiler og lægebiler at komme ind i område.   
Bus sluserne er en af trafikløsningsforslagene til løsningen på Nytorv/Østerågade og har til formål at sikre, at der ikke kører privatebiler på Nytorv/Østerågade. Løsning på problemet er at etablerer bus sluser tre steder. En af det kan være i Ved Stranden vej 10 meter før man kører ind på Østerågade. Den anden kan være på Nytorv under Aalborg biblioteket. Den tredje kan være på Østerågade retning fra Boulevarden lidt ind på Østerågade.  
Bus sluser bygges grundlæggende på to måder og har en pris omkring 0.5 mio stykket. 
Bus sluserne består ofte af et hul (en busgrav) eller en forhøjning (populært kaldet bundkarsknuser) i midten af vejbanen. Ved busgrave er bussernes hjul så langt fra hinanden, at de kan køre over hullet, mens personbiler vil falde i hullet. Ved bundkarsknuseren er forhøjningen oftest så høj at personbiler skraber på (deraf navnet) mens busserne har større frigang under bunden. Samtidigt er forhøjningen bred nok til at også dette giver bilerne passageproblemer. Forhøjninger er dog ved at gå af mode igen i takt med indkøb af lavgulvsbusser, der også kan få problemer med forhindringer i midten af vejen.
Ved Stranden vej retning til Østerågade kan bygge en bus sluse, som er afstand 10 meter til Østerågade. Privatbillister kan svinge før bus slusen og finde parkering til deres biler, og går på Østerågade område. 
I under Aalborg biblioteket har skilt markering at privatbiler og motorcykle var forbudt, men kører de alligevel igang med. Projektet giver en bus sluse til at unde biblioteket. Det kan undgå privatbiler og motorcykle meget virksom. Og privatbiler kan parks i Friis-parking, og kører ind på Nytorv parkering i Føtex og Salling.  
Østerågade kan begge en bus sluse i mellem på Østerågade. På hver side af slusen kan lave cykelsti ved Østerågade. Privatbiler kan svinge før Østerågade på Algade til parkering deres biler. 
Disse løsning forsøg på at gøre Nytorv/Østerågade delvist privatbilfri. Der er derfor kun tilladt for busser og cyklister. Det må være mere sikkerhed til fodgængere.
