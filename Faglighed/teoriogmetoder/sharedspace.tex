\subsection{Hvor anvendes Shared Space?}
\label{sub:hvor_anvendes_share_space}
Shared space anvendes oftest i områder med kryds, strækninger fra sted til sted, samt sammenhængende områder, såsom bycentre med smalle gader. Derudover anvendes shared space også i mindre landsbyer, tættere bymæssig bebyggelse, hvor der er blandede erhverv, og bymidter. Betingelsen for hvordan de ovennævnte områder fungerer som ‘shared space’, er ved, at der er lav hastighed, begrænset trafikmængder og balance mellem de forskellige trafikantgrupper, hvor ingen bestemt trafikantgruppe dominerer over de andre trafikgrupper. Shared space fungerer også som en alternativ vej, som afvikler trafikmængden af biler og cykler. Derfor anvendes shared space ikke ved trafikveje, cykelruter og bustrafik områder, da den ikke prioriterer en bestemt trafikgruppe i området. %(https://da.glosbe.com/da/sv/t%C3%A6ttere%20bebygget%20omr%C3%A5de 17/11-2015) og (http://vejregler.lovportaler.dk/showdoc.aspx?t=%2fV1%2fNavigation%2fTillidsmandssystemer%2fVejregler%2fAnlaegsplanlaegning%2fTrafikarealer+by%2fVejgeometri+i+byomrader%2f&docId=vd-shared-space-full 17/11-2015)
\subsection{Trafikmængder og hastigheder}
\label{sub:Trafikmaengder_og_hastgheder}
Det anbefales, at shared space anvendes ved trafikmængder på maks. 3.000- 4.000 motorkøretøjer pr. døgn. I tysk litteratur anbefales 4.000 motorkøretøjer, hvorimod engelsk litteratur anbefaler 2.000 motorkøretøjer med en hastighed på 30 km/t. Holland har områder, som er udformet med shared space, der afvikler 15.000 motorkøretøjer pr. døgn. Dette medvirker, at trafikken med store mængder trafik køretøjer fylder, og får fodgængere til at være ekstra opmærksomme på at krydse vejene. På Skvallertorget i den svenske by Norrköping har man erfaret, at mængden af fodgængere skal udgøre halvdelen, mere end de forventede antal biler. Nogenlunde samme antal cykler og biler skal være til stede, for at skabe balance i trafikken. Det anbefales, at kørende trafik i et shared space område bør udformes med en hastighed på 15-20km/t, hvilket svarer til en meget lav hastighed. %(http://vejregler.lovportaler.dk/showdoc.aspx?t=%2fV1%2fNavigation%2fTillidsmandssystemer%2fVejregler%2fAnlaegsplanlaegning%2fTrafikarealer+by%2fVejgeometri+i+byomrader%2f&docId=vd-shared-space-full 17/11-2015)
\subsection{Ens eller dobbeltrettet}
\label{sub:ens_eller_dobbeltrettet}
Shared space er udformet som dobbeltrettede gader. Dette resulterer i en bedre adgang til området uden at der skal foretages nogle omveje for trafikken, hvorimod ensrettet vej vil prioritere den ene kørselsretning. Dette medfører en højere hastighed og tavler, som er i strid med shared space princippet, der ligger til grund for at reducere hastigheden og afmærkningerne. I den anledning bør shared space altid være dobbeltrettet, da den også skal tage hensyn til cykler, der cykler i hver sin retning. %(http://vejregler.lovportaler.dk/showdoc.aspx?t=%2fV1%2fNavigation%2fTillidsmandssystemer%2fVejregler%2fAnlaegsplanlaegning%2fTrafikarealer+by%2fVejgeometri+i+byomrader%2f&docId=vd-shared-space-full 17/11-2015)

\subsection{Busser og parkering}
\label{sub:Busser_og_parkering}
Det anbefales ikke at føre bustrafik gennem et shared space område, men i særlige tilfælde kan det ske under korte strækninger.  Da hastigheden i shared space område er lav, kan bussen risikere at blive bremset eller forsinket af krydsende fodgængere. City- og servicebuslinjer kan føres gennem shared space områder, dog tager ruteplanlægningen højde for, at busserne forventer lav hastighed i området.
I et shared space område bør parkeringen være begrænset, og der skal være afmærkede båse. Derudover skal et shared space område tilbyde passende cykel parkeringsfaciliteter. %(http://vejregler.lovportaler.dk/showdoc.aspx?t=%2fV1%2fNavigation%2fTillidsmandssystemer%2fVejregler%2fAnlaegsplanlaegning%2fTrafikarealer+by%2fVejgeometri+i+byomrader%2f&docId=vd-shared-space-full 17/11-2015)

\subsection{Eksempler på Shared Space anvendelse}
\label{sub:eks_shared_space}
DER KOMMER BILLEDER - ENDNU IKKE LAVET
