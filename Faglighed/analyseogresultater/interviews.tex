
\subsection{Interviews}
Eftersom at fodgængere blev observeret som et problem for cyklisterne, så blev fodgængere også observeret grundigere. Der blev identificeret, at fodgængere også var delvis utryg ved Nytorv området, hvilket gav årsag til grundigere undersøgelse, med blandt andet interview med fodgængere.\footnote{XXXXXXX vej.lovportaler.dk} http://vejregler.lovportaler.dk/showdoc.aspx?t=%2fV1%2fNavigation%2fTillidsmandssystemer%2fVejregler%2fAnlaegsplanlaegning%2fTrafikarealer+by%2fVejgeometri+i+byomrader%2f&docId=vd-shared-space-full 10/11-2015}


\subsection{Anvendelse af interviews i praksis}
I denne rapport blev der foretaget et kvalitativt interview med forbigående fodgængere ved Nytorv/Østerågade i Aalborg. Det kvalitative interview leder til kvalitativ viden fra forskellige aspekter, og virker ofte som en metode for, at undersøge nogle bestemte forholde. Under det kvalitative interview bliver interviewpersonerne opfordret til at beskrive hvad de oplever, føler og hvilke ideer og holdninger de har til det givende emne. Derved opnås der under det kvalitative interview konkret viden, i stedet for generelle meninger, som er karakteristisk ved det kvantitative interview. \footnote{ Hans Reitz Forlag, interview introduktion til et håndværk 2. udgave, side 48}
\\

Herunder kunne et eksempel på en kvalitativ interview være et delvis strukturerede interview, som anvendes, når den teoretisk og praktisk viden er kendt på forhånd, men er åben for nye synsvinkler, holdninger og oplevelser, som interviewpersonerne deler. I forbindelse med interviewet bliver der belyst nogle forholde, som til en vis grad er udarbejdet på forhånd. I rapporten er der en påstand om at det kan føles utrygt, når trafikmiljøet er integreret nede ved Nytorv/Østerågade, herved bruges interviewene blandt andet, som belæg for at det kan virke utrygt, at færdes i området. \footnote{Ib Andersen, Den skinbarlige virkelighed, side 169}


\subsection{Formålene med interviewene ved Nytorv/Østerågade i Aalborg}
Formålet med interviewet er, at finde initierende problemer og løsninger til, hvordan der kan opnås en effektiv og innovativ sammenbinding af gågade systemet i Aalborg Nytorv/Østerågade. Der blev interviewet fodgængere og cykelister, med hensigten om at de skulle have medbestemmelse i, at skabe en bedre oplevelse af sikkerhed og tryghed i området.
\\

I den forbindelse blev der gennemført ca. 20 kvalitative interviews , hvor der blev talte med en forbigående fodgængere og cyklister ad gangen. Interviewene skal bidrage med, at finde løsninger til, hvordan der kan skabes et sikkert og trygt trafikmiljø for alle fodgængere og cyklister.
I værktøjskassen vil der være en række spørgsmål om, hvordan fodgænger og cyklister oplevelser er med at have et integreret trafikmiljø, ønsker ift. hvordan en effektiv sammenbinding af gågadesystemet i Aalborg kunne være, og hvad de tænker om forslaget om løsningerne til problemstillingen i området.
Interviewet er åbent undersøgende om fodgængernes og cyklisterne sikkerhed og tryghed. Der blev afsat 2 timer til interviewet den 14 oktober 2015 fra klokken 13 til 15.\footnote{https://innovation.blogs.ku.dk/metode/det-kvalitative-interview/ 27/10- 2015}

\subsection{Interview resultaterne med forbigående fodgængere og cyklister ved Nytorv/Østerågade i Aalborg}
 Ud fra de kvalitative interviews (se bilag  \cref{chap:interviews}) med forbigående fodgængere og cykelister, resulterede interviewene til, at der var en større enighed om hvorvidt forbipasserende fodgænger og cyklister følte sig utrygge ved at passere Nytorv/Østerågade området i forbindelse med, det integreret trafikmiljø der er. Eksempelvis påstod flere af interviewpersonerne, at de brugte en del tid på at holde øje med om cyklisterne havde set dem. Spørgsmålet lød på om de følte sig trygge ved at gå over fodgængerfeltet, når der færdes mange cyklister ude ved Nytorv/Østerågade i Aalborg, hertil svarede en del,\emph{“Egentlig ikke - jeg synes, at cyklerne kommer meget uventet og meget hurtigt. Jeg holder utroligt meget øje.”} Dette beviser blot at flere fodgængere føler sig utrygge ved at gå over fodgængerfeltet ved området, selvom at de har førsteret til at gå over.
\\

Der er flere årsager, til hvorfor fodgængerene og forbigående cyklister føler sig utrygge i området, eksempelvis mente interviewepersonerne,\emph{”(…), at der kan være kaos herude nogle gange, og det er både cyklister, biler og busser der skaber den kaos.”}. Det vil sige at årsagen til følelsen af utrygheden iblandt fodgængere og forbigående cyklister, delvis er det integreret trafikmiljø der eksisterer, som i visse tilfælde kan udløse et kaos, manglende hensynstagen til andre trafikgrupper, såsom cyklister som ikke holder for fodgængerne ved fodgængerfeltet, mindre kontrol over udelukkelse af privatbiler, da de kan fremføre til større skader, dog er der ingen af interviewpersoner der har oplevet ulykker,  eksempelvis mener en interviewperson, at før vedkommende kan føle sig tryg i trafikken kræver det,\emph{“(…) at busserne og bilerne holder tilbage. Det er dem der kan lave noget skade. Jeg har gået herinde rigtig meget, og jeg har aldrig set en ulykke.”}, selvom at ingen af interviewpersonerne har oplevet en ulykke i området, er tanken om at en større ulykke kan finde sted alligevel skaber utryghed. Derudover skaber den knibende vejplads til cykelister også en utryghed for fodgængerne, eksempelvis påstod en interviewperson, at \emph{“(…) på Boulevarden lagde jeg mærke til at cyklisterne ikke kunne være der, så de måtte trække deres cykler på fortovet (…)".}, da cyklisterne i vissetilfælde trækker deres cykler på fortovet af manglende plads til dem ved Nytorv/Østerågade, skaber det en utryghed for fodgængerne, der går på fortovet.
\\

Der var også en enighed iblandt de forbipasserende fodgængerne og cyklisterne om, at der ønskes en differentiering mellem cyklisterne og kørebanerne. Ønsket lød på, at \emph{“(…) lede cyklerne uden om Nytorv eller dele cyklisterne ud fra kørebanen(…)"}. I og med, hvis ønsket om et differentierede trafikområde blev vedtaget bevæger Nytorv/ Østerågade sig længere væk fra shared space konceptet, hvilket i vissetilfælde vil virke mere trygt for alle trafikantgrupper.
I forbindelse med spørgsmålet om, hvilke ønsker eller forslag de forbigående fodgængere og cykelister havde til at binde gågaden sammen, blev der eksempelvis forslået nogle ønsker om, at \emph{”(…)man kan male fodgængerfeltet hvidt.”}. På nuværende tidspunkt er fodgængerfeltet mellem Bispensgade og Lille Kongensgade grå, en hvid fodgængerfelt frem for den grå, vil virke mere synlig for de kørende og cyklende trafikantgrupper. Nogle interviewpersoner mener også, at mere kontrol over udelukkelse af privatkørsel, en gennemvej for cykelister der skal mod universitet, undergang, overgang eller tunnel vil skabe mere tryghed i området. I den anledning var der også flere ønsker om en køretøjsfrizone, dog var der i højere grad enighed om, at cyklisterne ikke skulle udelukkes fra Nytorv/Østerågade, da nogle synes at de skaber liv i området, og andre påstod at det ikke ville kunne lade sig gøre med den store mængde af cyklister, der cykler til universitet og andre steder hver dag.

\footnote{Fatimah Daoud,æbleskiver}
