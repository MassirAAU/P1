\subsection{Første observation}
\label{sub:foerste_obs}
Aalborg er byen som arbejder på, at skabe en cykelby.%(http://www.aalborgcykelby.dk/aalborg-cykelby)
Som en cykelby kræver det, at der er en optimal sikkerhed for cyklister blandt andre trafikantgrupper. Trafik Flowet ved åbningen af Nytorv blev observeret, området ved McDonalds og Burger King, hvor der kunne ses at de fleste cyklister havde nogle mindre alvorlige konflikter med bilerne der krydser Nytorv illegalt. Tværtimod havde cyklisterne lidt større konflikter med fodgængere samt busserne, hvilket også gav et stop for trafikflowet. Derudover kunne der ikke observeres en tryghed på cyklisternes ansigtsudtryk, eftersom de hele tiden skulle være fokuseret og parat til, at bremse ned for fodgængere eller busser. En mindre vurdering kunne bekræfte, at cyklisterne ikke havde en optimal tryghed ved området pga. fodgængere og delvist busserne, eftersom busserne ikke hele tiden krydser Nytorv, men blot hver 5.-6. minut.

Trafik flowet ved slutningen af Nytorv blev observeret, området ved Salling og Friis, her kunne der ses konflikter blandt bilister og cyklister, eftersom cyklisternes cykelsti bliver fjernet. Der kunne tydelig ses en utryghed blandt cyklisterne og bilisterne, da trafikantgrupperne delte vejbanen, hvilket skabte ustruktureret trafikflow ved området og nogle ulovlige overhalinger.

Eftersom fodgængere blev observeret som et problem for cyklisterne, så blev fodgængere også observeret grundigere. Der blev identificeret, at fodgængere også var delvis utryg ved Nytorv området, hvilket gav årsag til grundigere undersøgelse, såsom interview med fodgængere.

\subsection{Dybbere undersøgelse af første observation}
\label{sub:def_konflikt}
I første omgang blev der blot observaret, hvor der blev antaget en række
\subsection{Resultater af observation}
\label{sub:def_konflikt}
BILLEDE AF OBSERVATIONSOMRÅDET (TA)
Som der kan ses på billedet, blev området i første omgang målt op fra kanten som er markeret med farven rød til fodgængere feltet og det samme fra den anden kant som er markeret med farven blåt. Bemærk de andre markeringer er forskellige dimensioner som bruges til beregning af TA-værdien, altså i TA-værdi formlen: ta=d/V ligningen skal skrives HER
Metoden bag TA-værdi var, at der blev lavet nogle målinger, dog blev de forskellige distancer noteret og derudover blev der også taget tid til hver enkelt konflikt. I første omgang blev tiderne antaget, som vises i tabel x:
(1) Fart: 6 km/t TA:1.5s. Distance: 2m
(2) Fart: 10 km/t TA: 0.7s. Distance: 0.5-1m
(3) Fart: 4.6 km/t TA: 1.5s Distance: 2m
(4) Fart: 11 km/t TA: 0.4s Distance: 1m
(5) Fart: 10 km/t TA: 1s Distance: 2m
(6) Fart: 9 km/t TA: 0.1 Distance: 0.5m
(7) Fart: 12 km/t TA: 0.6 Distance: 0.5m
(8) Fart: 10km/t TA: 0.7 Distance 1m
(9) Fart 14 km/t TA: 0.4 Distance 0.5m
(10) Fart 8 km/t TA: 1 Distance 2m
(11) Fart 12 km/t TA: 0.3 Distance 1m
(12) Fart 11 km/t TA: 0.6 Distance 2m
(13) Fart 10 km/t TA: 0.8 Distance 1m
(14) Fart 6 km/t TA: 1 Distance 2m
(15) Fart 9 km/t TA: 0.9 Distance 1m
(16) Fart 10 km/t TA: 0.5 Distance 1m
(17) Fart 14 km/t TA: 0.3 Distance 0.5m
(18) Fart 12 km/t TA: 0.6 Distance 0.5m
(19) Fart 7 km/t TA: 1 Distance 2m
(20) Fart 5 km/t TA: 1.5 Distance 2m

Som tidligere nævnt, så blev distancen for hvert konflikt noteret, hvilket gøre det muligt at beregne en mere præcis TA-værdi. Tabel x, viser den beregnede TA-værdi for den bestemt hastighed samt distance som blev målt under udførelsen af metoden.
(1) Fart: 6 km/t TA:1.5s. Distance: 2m - V= 1.6m/s og d= 2 TA- værdi = 1.25s
(2) Fart: 10 km/t TA: 0.7s. Distance: 0.5-1m - V= 2.8m/s og d= 0.5m TA- værdi = 0.1s
(3) Fart: 4.6 km/t TA: 1.5s Distance: 2m - V= 1.3m/s og d= 2 TA- værdi = 1.5s
(4) Fart: 11 km/t TA: 0.4s Distance: 1m - V= 3m/s og d= 1 TA- værdi = 0.3s
(5) Fart: 10 km/t TA: 1s Distance: 2m - V= 2.8m/s og d= 2 TA- værdi = 0.7s
(6) Fart: 9 km/t TA: 0.1 Distance: 0.5m - V= 2.5m/s og d= 0.5 TA- værdi = 0.2s
(7) Fart: 12 km/t TA: 0.6 Distance: 0.5m - V= 3.3m/s og d= 0.5 TA- værdi = 0.1s
(8) Fart: 10km/t TA: 0.7 Distance 1m - V= 2.8m/s og d= 1 TA- værdi = 0.3s
(9) Fart 14 km/t TA: 0.4 Distance 0.5m- V= 3.9m/s og d= 0.5 TA- værdi = 0.13s
(10) Fart 8 km/t TA: 1 Distance 2m - V= 2.2m/s og d= 2 TA- værdi = 0.9s
(11) Fart 12 km/t TA: 0.4 Distance 1m - V= 3.3m/s og d= 1 TA- værdi = 0.3s
(12) Fart 11 km/t TA: 0.6 Distance 2m - V= 3.1m/s og d= 2 TA- værdi = 0.6s
(13) Fart 10 km/t TA: 0.8 Distance 1m - V= 2.8m/s og d= 1 TA- værdi = 0.3s
(14) Fart 6 km/t TA: 1 Distance 2m - V= 1.6m/s og d= 2 TA- værdi = 1.25s
(15) Fart 9 km/t TA: 0.9 Distance 1m - V= 2.5m/s og d= 1 TA- værdi = 0.4s
(16) Fart 10 km/t TA: 0.5 Distance 1m - V= 2.8m/s og d= 1 TA- værdi = 0.3s
(17) Fart 14 km/t TA: 0.3 Distance 0.5m - V= 3.9m/s og d= 0.5 TA- værdi = 0.13s
(18) Fart 12 km/t TA: 0.6 Distance 0.5m - V= 3.3m/s og d= 1 TA- værdi = 0.1s
(19) Fart 7 km/t TA: 1 Distance 2m - V= 1.9m/s og d= 2 TA- værdi = 1s
(20) Fart 5 km/t TA: 1.5 Distance 2m - V= 1.4m/s og d= 2 TA- værdi = 1.4s
(Beregnet TA-værdi skal sættes i tabel)

Som beskrevet i afsnit X, så kan grafen for TA-værdi bruges til, at identificere om hvor alvorlig konflikterne egentlig er ved området. I grafen kan der ses, hvor de beregnede TA-værdier er placeret, altså om det tilhører alvorlig konflikt eller ej.
(GRAF skal laves, hvor der er en tildeling af konfliktgraderne, herunder skal beregnede TA-værdier sættes ind)
Som der kan ses på grafen, så er størst delen af konflikterne ret alvorlige eftersom, trafikanterne kommer ret tætte på hinanden i nogle bestemte hastigheder.
Resultaterne fra TA-værdi og adfærdsregistering sættes sammen: (HER SKAL TABELLEN SÆTTES FRA ADFÆRDregisI)
Der var en hypotese om, at hvis antallet af tidlige samspil stiger og antallet af sene samspil falder. så vil trafiksikkerheden og trygheden fra området blive forøget. Ses der på adfærdsregistering tabellen, så kan der observeres at resultaterne er det omvendte af hypotesen, så det vil sige at sikkerheden samt trygheden bliver reduceret ved området. Udefra TA-værdien kunne der også bekræftes, at sikkerheden og trygheden ikke er helt optimalt.
Udefra TA-grafen, så kan der ses, at 7 ud af 20 konflikter ikke er alvorlige, hvoraf 13 ud af 20 er alvorlige. Det svar til, at 65% procent af konflikterne er alvorlige, hvoraf resterne 35% er mindre alvorlige som ikke har det stor betydning ved områdets sikkerhed.
