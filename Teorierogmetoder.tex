%mainfile: master.tex
\chapter{Teorier og metoder}
\label{chap:teorier_og_metoder}
I dette afsnit bliver der set på de relavante teorier og metoder der bliver anvendt til problematikken ved Nytrov. Herunder bliver der givet en introduktion til hver emne, før det bliver præsenteret i rapporten.
\newpage
\section{Shared Space}
\label{sec:shared_space}

\section{Observationsmetoder}
\label{sec:observationsmetoder}

\section{Trafiktællinger}
\label{sec:trafiktaellinger}

Dette afsnit er inspireret af kilden %http://vej08.vd.dk/mastra/mastradok/dok/TrafiktaellingerPlanUdfoerEfterb.pdf
\\
Trafiktællinger udføres til mange formål. Det gælder alt fra at kontrollere den overordnede vejplanlægning
til at undersøge klagesager om for høje trafik hastigheder. Trafiktællinger bruges bl.a. til at finde løsninger
til opgaver omhandlende trafiksikkerhed, kapacitet og miljøforhold, samt statistikker over trafikudviklingen
og hastigheden af vejnettet. Man kan foretager trafiktællinger manuelt eller ved brug af maskiner.
Manuelle tællinger fungerer ved at personer registrere trafikken på det pågældende sted, ofte ved hjælp af
tælleblokke, håndtællere eller håndterminaler. (Mere om manuelle tællinger \cref{sec:manueltaelling}). Ved maskine
tælling fortages registreringerne automatisk ved brug af et tælleapparat, hvor mennesker ikke medvirker.
\subsection{Manuel Tælling}
\label{subs:manueltaelling}

Manuel tælling er som sagt, hvor det er mennesker der tæller trafikken. Manuel tælling er en god metode,
når der ønskes at kende trafikkens specifikke trafikstrømme. Et typisk forløb for manuel tælling er opbygget
op af 6 trin.
\\\\
1) Som det første skal formålet for tællingen bestemmes, samt hvilken resultat type, som ønskes af opnå.
\\\\
2) Der skal besluttes placeringen af tælleposterne, hvor der skal tages hensyn til at tælleren ikke genere
trafikken. Tællerne skal også have frit udsyn for parkerende biler, buskaser og lignende under hele
tælleperioden. Man skal derfor overveje om forholdene kan ændre sig undervejs.
\\\\
3) En af de betydelige usikkerheder ved trafiktællinger, er valget af tælleperioden. Der kan være meget stor
variation fra dag til dag og time til time, hvis man. ønsker at finde årsdøgntrafikken. Der er f.eks. stor forskel
på trafikmængden i weekenden kontra hverdage og myldertiden om morgen og eftermiddagen kontra
andre midt på dagen og om aftenen og natten. Manuelle tællinger vare typisk 4, 6 eller 12 timer og
sjældent et helt døgn.
\\\\
4) Når tælleposterne og tidspunkterne er fastlagt, bestemmes antallet af tællere til posterne, efter
trafikmængden ved stedet. Ifølge vejdirektoratet er kvaliteten af resultaterne afhænger af antal tællere. Er
tælleposterne underbemandet, vil resultaterne blive uanvendelige. Hvis man er uvidende om
trafikmængden, og dermed antallet af tæller som er nødvendige, kan man foretage en prøvetælling inden.
Der skal også bestemmes valget af tælleudstyr. Ved mindre trafikmængder kan der anvendes blyant og papir, ved større trafikmængder anvendes ofte håndtællere, håndterminaler eller tællepulte.
En håndtæller består typisk af et apparat med en knap, som tæller 1 frem, hver gang der trykkes, og en anden knap som nulstiller apparatet igen. Håndterminaler –og pulte, er et mere avanceret værktøj. Apparatet kan tælle flere trafikgrupper af gangen, eller hvilken retning trafikken kommer fra, på både snitlængde og på krydstællinger.
%Billeder skal sættes ind fra word dokumnetet
\\\\
5) Inden tællerne begynder, er det vigtigt at de er sat sig ind i overstående bestemmelser for tælleforløbet, således der ikke opstår tvivler undervejs. Ligeledes udarbejdes et tælleskema inden tællingen påbegyndes.
\\\\
6) Resultatbehandling af tællingen.
