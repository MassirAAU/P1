%mainfile: master.tex
\chapter{Problemformulering}
\label{chap:problemformulering}


\\\\
Nytorv/Østerågade er i centrum af Aalborg by. Der er mange moderne butikker, kaffebarer, restauranter og Aalborg`s hovedbibliotek, som giver bosiddende og besøgende et sted at nyde deres fritid. Der er derfor mange mennesker som går ture, shopper, drikker kaffe og spiser sammen med venner i området på Nytorv/Østerågade. Aalborg centrum er et hyggeligt sted at være og med meget leben. Ifølge indledningen kan projektet se, at Aalborgs befolkning stiger. Nytorv/Østerågade er blevet trafiktalt og det har vist sig, at trafikmængden stiger år efter år. Det drejer sig om både busser, cyklister og fodgængere. Der bliver trafikkonflikter med hensyn til shared space. 
\\\\
Hvilke konflikter finder sted mellem de forskellige trafikantgrupper på Nytorv/Østerågade?
\\
Hvordan kan en ny trafikplanlægning løse konflikterne?
\\
Hvordan skabes et harmonisk trafik område på Nytorv/Østerågade? 
\\\\
Projektet vil komme frem til løsningsforslag disse spørgsmål.



\section{Afgrænsning}
\label{sec:afgraensning}

I dette projekt er der fokuseret på trafikkonflikter mellem de forskellige trafikantgrupper på Nytorv/Østerågade i Aalborg. Fokuspunktet er fodgængernes tryghed overfor cyklisterne og bilerne. Der vil altså ikke blive fokuseret på buschauførrenes syn på sikkerheden eller dem der kører privatbiler, da de desuden er udelukket fra området. Der er blevet valgt i projektet at afgrænse sig til at lave kvalitative interviews af fodgængere og observationer af Nytorv. Desuden er der en afgrænsning i forhold til de forskellige trafikantgrupper, da der vil blive lavet interviews, udelukkende med forbipasserende som går eller cykler på Nytorv. Der vil blive undersøgt, om fodgængerne føler sig trygge, når de går på Nytorv/Østerågade i Aalborg. Der tages udgangspunkt i at undersøge hvilket problem det er, og hvordan man i området kan skabe et bedre trafikflow.

I undersøgelsen har projektet som formål at vurderer disse trafikkonflikter, og ved hjælp af vores observationsmetoder og interviews, komme med et eller flere forslag til nogle yderst konstruktive løsningsforslag til Nytorv/Østerågade området.  Dette projekt vil undersøge, hvordan det er bedst muligt at give sikkerhed og tryghed til menneskerne som besøger Nytorv/Østrågade i Aalborg.
