%mainfile: master.tex
\chapter{Afgrænsning}
\label{chap:Afgrænsning}

<<<<<<< HEAD
Afgrænsning
I dette projekt er der fokuseret på trafikkonflikter mellem de forskellige trafikantgrupper på Nytorv/Østerågade i Aalborg. Fokuspunktet er fodgængernes tryghed overfor cyklisterne og bilerne. Der vil altså ikke blive fokuseret på buschauførrenes syn på sikkerheden eller dem der kører privatbiler, da de desuden er udelukket fra området. Der er blevet valgt i projektet at afgrænse sig til at lave kvalitative interviews af fodgængere og observationer af Nytorv. Desuden er der en afgrænsning i forhold til de forskellige trafikantgrupper, da der vil blive lavet interviews, udelukkende med forbipasserende som går eller cykler på Nytorv. Der vil blive undersøgt, om fodgængerne føler sig trygge, når de går på Nytorv/Østerågade i Aalborg. Der tages udgangspunkt i at undersøge hvilket problem det er, og hvordan man i området kan skabe et bedre trafikflow.

I undersøgelsen har projektet som formål at vurderer disse trafikkonflikter, og ved hjælp af vores observationsmetoder og interviews, komme med et eller flere forslag til nogle yderst konstruktive løsningsforslag til Nytorv/Østerågade området.  Dette projekt vil undersøge, hvordan det er bedst muligt at give sikkerhed og tryghed til menneskerne som besøger Nytorv/Østrågade i Aalborg.
=======



Dette projekt har et brændpunkt, nemlig trafikmængder mellem forskellige trafikantgrupper på Nytorv/Østerågade i Aalborg. Der er blevet valgt i projektet at afgrænse sig til at lave observation af biler og kvalitative interviews med fodgængere og forbipasserende cyklister på Nytorv. Desuden er der en afgrænsning i forhold til forskellige trafikantgrupper, da der vil blive interviewet med forbipasserende, som går eller cykler på Nytorv. Der vil blive undersøgt menneskes følelser om sikkerhed og tryghed, når de går på Nytorv/Østerågade i Aalborg. Der tages udgangspunkt i at undersøge hvilke problem det er, og hvordan området i Nytorv/Østerågade kan undgå trafikmæssigt i forskellige trafikantgrupper.



I denne undersøgelse fokuserer projektet på trafikkonflikter mellem fodgængere, cyklister og biler på Nytorv/Østerågade området. Projektet analyserer og vurderer disse trafikkonflikter, som afhænger af trafikteorier og metoder prøve at komme med forslag til nogle yderst konstruktive med forskellige løsningsmodeller til Nytorv/Østrågades trafikkonflikter.  Dette projekt vil se hvordan det er bedst muligt, at give sikkerhed og tryghed til fodgængere og cyklister, som besøger Nytorv/Østrågade i Aalborg.
>>>>>>> 4b8eedd628523c7f432511ac85a04540c3c87692
