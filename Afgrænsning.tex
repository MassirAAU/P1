%mainfile: master.tex
\chapter{Afgrænsning}
\label{chap:Afgrænsning}

Afgrænsning 
Dette projekt har et brændpunkt, nemlig trafikkonflikter mellem forskellige trafikantgrupper på Nytorv/Østerågade i Aalborg. Der er blevet valgt i projektet at afgrænse sig til at lave kvalitative interviews med fodgængere og observation af Nytorv. Desuden er der en afgrænsning i forhold til forskellige trafikantgrupper, da der vil blive interviewet med forbipasserende, som går eller cykler på Nytorv. Der vil blive undersøgt menneskes følelser om sikkerhed og tryghed, når de går på Nytorv/Østerågade i Aalborg. Der tages udgangspunkt i at undersøge hvilke problem det er, og hvordan området i Nytorv/Østerågade kan undgå trafikmængder i forskellige trafikantgrupper. 

I denne undersøgelse har projektet fokuseret på trafikkonflikter i Nytorv/Østrågade i Aalborg. Projektet analyserer og vurderer disse trafikkonflikter, som afhænger af trafikteorier og metoder prøve at komme med forslag til nogle yderst konstruktive med forskellige løsningsmodeller til Nytorv/Østrågades trafikkonflikter.  Dette projekt vil se hvordan det er bedst muligt, at give sikkerhed og tryghed til menneskerne som besøger Nytorv/Østrågade i Aalborg. 
